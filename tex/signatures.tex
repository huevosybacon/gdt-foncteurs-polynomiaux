\newcommand{\PolyEnd}{{\mathrm{\cP olyEnd}}}
\newcommand{\PolyFun}{{\mathrm{\cP olyFun}}}
\newcommand{\PolyMnd}{{\mathrm{\cP olyMnd}}}

\section{Signatures}
\section{Composition of polynomial functors as signatures}

\section{Polynomial endofunctors and term algebras}
For $I\in\Set$, the category $\PolyEnd_I$ of polynomial endofunctors on $Set/I$
is defined to be the category $\PolyFun_{I,I}$.

\subsection{Term algebra of a polynomial endofunctor}
Let $P\in\PolyEnd_I$ be a polynomial functor written as
\[
    I\xot{s} E\xto{p} B\xto{t} I.
\]
Then the term algebra

\subsection{Free algebras of endofunctors}
Let $\cC$ be a category and let $F\in [\cC,\cC]$ be an endofunctor on $\cC$.
Then an \emph{$F$-algebra} is defined to be a morphism $FX\to X$ in $\cC$, and
the category $F\alg$ of $F$-algebras is defined to be the (non-full!) subcategory of
$\cC\arr$ whose objects are the $F$-algebras, and whose morphisms are commuting
squares
\[
    \xymatrix{
      FX \ar[d]\ar[r]^{Ff}
      &FY\ar[d]\\
      X\ar[r]^f
      &Y.
    }
\]
The codomain functor will be denoted
$U_F:F\alg\to\cC$.

\begin{theorem} \label{thm:free-monad}
    TFAE:
    \begin{enumerate}
    \item For every $F\in[\cC,\cC]$, the functor $U_F$ has a left adjoint (the
        free-algebra functor).
    \item The forgetful functor $\Mnd(\cC)\to[\cC,\cC]$ has a left adjoint (the
        free-monad functor).
    \end{enumerate}
\end{theorem}

The theorem essentially follows from the following fundamental observations.
\begin{proposition}
    Let $F\in\fun \cC \cC$ be such that $U_F$ has a left adjoint $L_F$. Then the
    monad $U_FL_F$ is the free monad on the endofunctor $F$.
\end{proposition}

\begin{proposition}
    Let $F\in\fun \cC \cC$ and let $\bar{F}$ be the free monad on $F$. Then
    there is an equivalence of categories $F\alg\simeq\bar{F}\alg$, where
    $\bar{F}\alg$ is the Eilenberg-Moore category of algebras of the monad
    $\bar{F}$.
\end{proposition}
\subsection{Free algebras of polynomial endofunctors}
Let $P$ be a finitary (i.e. the fibres of $p$ are finite) polynomial
endofunctor, as shown.
\[
    I\xot{s} E\xto{p} B\xto{t} I
\]
Then the function symbols (operations) of the signature
$\underline{P}=(B,I,I,p\inv,s,t)$ can be \emph{composed} in a meaningful way,
since the set of output types is the same as the set of input types. Indeed, if
we denote $P^n:=P\circ P\circ\ldots\circ P \ (n \text{ times})$, then it is
clear that the operations of $P^{k+1}$ are trees of the form $b(f_e)_{e\in p\inv(b)}$,
where $b\in B$ is an operation of $P$ and each $f_e$ is an operation of $P^k$
with output type $s(e)$. Thus, for $n\geq 0$, operations of $P^n$ are finite
trees of operations of $P$ whose leaves are all at height $n+1$.

Note that the sets of operations of the signatures $\underline{P^n}$ have in
general a non-trivial intersection, as seen by the following exercise (which has
a simple proof by induction on $n$.)
\begin{exercise} Show that
        $\{ T\in P^n \} \cap \{ T\in P^{n+1} \} = \{ T\in P^n \ssep T \text{ has
          no leaves} \}.$
\end{exercise}

This information is captured in the definition of the \emph{term algebra} on a
signature.
\begin{definition}
    Let $S=(B,I,I,p\inv,s,t)$ be a signature. Then the \emph{term algebra} of
    $S$ is the set $S^*$ of trees inductively defined by the grammar
    \[
        (i\in I,b\in B) t ::= x^i \ssep b(t_e)_{e\in p\inv b}
    \]
    where each $x^i$ is a \emph{variable of type $i\in I$}.

    A \emph{closed term} is an element of $S^*$ that does not contain any variables.
\end{definition}

\begin{exercise}
    Show that $(\underline{P})^*=\bigcup_{n\in\NN} \{ T\in P^n \}$.
\end{exercise}


Now, we can define $P^{\moo}$ to be the filtered union (the directed colimit)
\[
    P^\moo := \bigcup_{n\in\NN} \{ T\in P^n \ssep T \text{ has no leaves} \} =
    \bigcap_{n\in\NN} \{ T\in P^n \}.
\]
$P^\moo$ is canonically an element of $\Set/I$, with the function
$P^\moo\xto{t}I$ sending a tree to its output type. Clearly, $P^\moo$ is
non-empty if and only if $P$ has operations of arity $0$.

\begin{exercise}
    Show that $P^\moo$ is precisely the set of closed terms in $(\underline{P})^*$.
\end{exercise}

For $A\xto{f} I \in \Set/I$, we define $P_A$ to be the polynomial endofunctor
\[
    I\xot{s} E\xto{p+\emptyset} B+A \xto{(t,f)} I.
\]
It is easy to see that the signature $\underline{P_A}$ is obtained by adding an
operation of arity $0$ to the signature $\underline{P}$ for each element of $A$,
with output type given by $f$.

\begin{exercise}
    For $A\xto{f} I$, let $A$ denote the polynomial endofunctor
    $I\ot\emptyset\to A\xto{f} I$. Show that $P_A$ is the coproduct $P+A$ in
    $\PolyEnd_I$.
\end{exercise}

\begin{lemma} \label{lemma:free-algebra}
    For all $A\xto{f}I$, $P_A^\moo\xto{t}I$ is the free $P$-algebra on
    $A\xto{f}I$.
\end{lemma}
\begin{proof}
    An element of $P(P_A^\moo)$ is a formal composite $b(x_i)_{i\in p\inv b}$ of
    an operation $b$ of $P$ along with an element $x_i\in P_A^\moo$ for each
    input of $b$. But this is clearly once again an element of $P_A^\moo$, i.e.
    a leafless tree of operations of $P_A$. Thus we have a canonical function
    $P(P_A^\moo)\xto{\alpha_A} P_A^\moo$ exhibiting $P_A^\moo$ as a $P$-algebra.
    It is easy to see that $\alpha_A$ is an inclusion, whose complement is just
    the subset $A\subset P_A^\moo$.

    Now if $P(X)\xto{\alpha} X$ is any $P$-algebra, then a function $g:A\to X$
    gives rise to an algebra morphism $\bar{g}:P_A^\moo\to X$ defined
    inductively as
    \begin{align*}
      \bar{g} :  \quad a &\mapsto g(a),\quad a\in A \\
      b(x_i) &\mapsto \alpha(b(\bar{g}(x_i)))
    \end{align*}
    and it is easy to see that for every morphism $h\in P\alg(P_A^\moo,X)$,
    $h=\overline{h\restr A}$. The function $(\bar{\ }):Set/I(A,X)\to
    P\alg(P_A^\moo,X)$ is thus injective and surjective.
\end{proof}

\begin{corollary}
    $P^\moo$ is the initial object of $P\alg$.
\end{corollary}
\begin{proof}
    Clearly, $P^\moo=P_\emptyset^\moo$, which is the free algebra on
    $\emptyset\to I$. The free-algebra functor is a left adjoint, and we
    conclude using the previous lemma, since $\emptyset\to I$ is initial in
    $\Set/I$.
\end{proof}


For a more conceptual explanation, note that lemma \ref{lemma:free-algebra} is
equivalent to the following exercise.
\begin{exercise}
    For $P\in\PolyEnd_I$ and $A\xto{f} I$, we have the equivalence of categories
    $P_A\alg\simeq P_A^\moo/P\alg$, where $P_A^\moo/P\alg$ denotes the coslice
    category under $P_A^\moo$.
\end{exercise}

\subsection{The ``free-polynomial-monad'' monad}
Let the category of \emph{polynomial monads} $\PolyMnd_I$ be the category of
monoids in the monoidal category $\PolyEnd_I$. Its objects are therefore monads
on $\Set/I$ whose endofunctor is polynomial, and whose unit and multiplication
are cartesian natural transformations, and we have the obvious square of
forgetful functors
\[
    \xymatrix{
      \PolyEnd_I\ar[d]
      &\PolyMnd_I\ar[l]\ar[d]\\
      [\Set/I,\Set/I]
      &\Mnd(Set/I).\ar[l]
    }
\]

Theorem \ref{thm:free-monad} now tells us that the endofunctor
$P^*:A\mapsto P_{(A)}^\moo$ is the free monad on the polynomial endofunctor $P$.
It is easy to show that the construction is functorial on $P$, and therefore
defines a functor $(-)^*:\PolyEnd_I\to\Mnd(Set/I)$
This subsection is devoted to showing that the functor $(-)^*$ factors through
$\PolyMnd_I$, and thus provides a left adjoint to the forgetful functor
$\PolyMnd_I\to\PolyEnd_I$, living above $\Mnd(\Set/I)\to[\Set/I,\Set/I]$. 

\begin{lemma}
    Let $S=$
\end{lemma}

\subsection{Term algebras}



%%% Local Variables:
%%% mode: latex
%%% TeX-master: "../notes"
%%% End:
