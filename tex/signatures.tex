\newcommand{\PolyEnd}{{\mathrm{\cP olyEnd}}}
\newcommand{\PolyFun}{{\mathrm{\cP olyFun}}}
\newcommand{\PolyMnd}{{\mathrm{\cP olyMnd}}}

\section{Signatures}
\section{Composition of polynomial functors as signatures}

\section{Polynomial endofunctors and term algebras}
For $I\in\Set$, the category $\PolyEnd_I$ of polynomial endofunctors on $Set/I$
is defined to be the category $\PolyFun_{I,I}$.

The category $\PolyEnd$ is defined to be the category whose objects are the $P \in
\PolyEnd_I$ (for all $I \in \Set$) and whose morphisms are morphisms of polynomial
functors of the form below.
\begin{displaymath}
  \begin{tikzcd}
    I_P \arrow[d,"f_0"']
    &E_P % \pullbackcorner
    \arrow[l] \arrow[r] \arrow[d,"f_2"']
    &B_P \ar[r] \ar[d,"f_1"']
    &I_P \ar[d,"f_0"]
    \\
    I_Q
    &E_Q \ar[l] \ar[r]
    &B_Q \ar[r]
    &I_Q
  \end{tikzcd}
\end{displaymath}
(Note that a morphism of polynomial \emph{endo}functors is more than just a
morphism of polynomial functors, since the rightmost and leftmost vertical
morphisms are the same.) For any $I \in \Set$, $\PolyEnd_I$ is thus a full
subcategory of $\PolyEnd$.

\subsection{Free algebras of endofunctors}
\begin{notation}
  For categories $\cC, \cD$, we denote the category of functors by $\fun \cC
  xy pic vs tikz cd\cD$.
\end{notation}

Let $\cC$ be a category and let $F\in [\cC,\cC]$ be an endofunctor on $\cC$.
Then an \emph{$F$-algebra} is defined to be a morphism $FX\to X$ in $\cC$, and
the category $F\alg$ of $F$-algebras is defined to be the (non-full!)
subcategory of $\cC\arr$ whose objects are the $F$-algebras, and whose morphisms
are commutative squares
\begin{displaymath}
  \begin{tikzcd}
    FX \ar[d]\ar[r,"Ff"]
    &FY\ar[d]
    \\
    X\ar[r,"f"]
    &Y.
  \end{tikzcd}
\end{displaymath}
The (forgetful) codomain functor will be denoted $U_F:F\alg\to\cC$.

We denote the category of monads (and monad morphisms) on $\cC$ by $\Mnd(\cC)$.
Note that this is just the category of monoids in the monoidal category $\fun
\cC \cC$. There is an obvious forgetful functor $\Mnd(\cC) \to \fun \cC \cC$.

\begin{theorem} \label{thm:free-monad} TFAE (The following are equivalent):
  \begin{enumerate}
  \item For every $F\in[\cC,\cC]$, the functor $U_F$ has a left adjoint (the
    free-algebra functor).
  \item The forgetful functor $\Mnd(\cC)\to[\cC,\cC]$ has a left adjoint (the
    free-monad functor).
  \end{enumerate}
\end{theorem}

The theorem essentially follows from the following fundamental propositions.
\begin{proposition}
    Let $F\in\fun \cC \cC$ be such that $U_F$ has a left adjoint $L_F$. Then the
    monad $U_FL_F$ is the free monad on the endofunctor $F$.
\end{proposition}

\begin{proposition}
  Let $F\in\fun \cC \cC$ and let $\bar{F}$ be the free monad on $F$. Then there
  is an equivalence of categories $F\alg\simeq\bar{F}\alg$, where $\bar{F}\alg$
  is the Eilenberg-Moore category of algebras of the monad $\bar{F}$.
\end{proposition}
\subsection{Free algebras of polynomial endofunctors}
Let $P$ be a finitary (i.e. the fibres of $p$ are finite) polynomial
endofunctor, as shown.
\[
  I\xot{s} E\xto{p} B\xto{t} I
\]
We denote $P^n:=P\circ P\circ\ldots\circ P \ (n \text{ times})$ for $n \geq 1$,
and define $P^0$ to be the identity polynomial endofunctor $I\xot{id} I \xto{id}
I \xto{id} I$.

Indeed, since $P^{k+1} = P \circ P^k$, the operations of $P^{k+1}$ are by
definition trees of the form $b(f_e)_{e\in p\inv(b)}$, where $b\in B$ is an
operation of $P$ and each $f_e$ is an operation of $P^k$ with output type
$s(e)$. Thus, for $n\geq 0$, operations of $P^n$ are finite trees of operations
of $P$ whose leaves are all at height $n+1$. We will abuse notation and write
$T\in P^n$ for $T$ an operation of $P^n$.

Note that the sets of operations of $P^n$ have in general a non-trivial
intersection, as seen by the following exercise.
\begin{exercise} Show that $\{ T\in P^n \} \cap \{ T\in P^{n+1} \} = \{ T\in P^n
  \ssep T \text{ has no leaves} \}.$
\end{exercise}
\begin{proof}
  $P^{n+1} = P^n \circ P$; thus the operations of $P^{n+1}$ are of the form
  $T(b_e)_e$ where $T$ is an operation of $P^n$ and $(b_e)_e$ is a list of
  operations of $P$ indexed by the leaves of $T$.
\end{proof}


Now, we define $P^{\moo}$ to be the filtered union (the directed colimit)
\[
  P^\moo := \bigcup_{n\in\NN} \{ T\in P^n \ssep T \text{ has no leaves} \} =
  \bigcup_{n\in\NN} \{ T\in P^n \} \cap \{ T \in P^{n+1} \}.
\]
The function $P^\moo \xto{t} I$ sending an operation to its output type
canonically exhibits $P^\moo$ as an element of $\Set/I$. Clearly, $P^\moo$ is
non-empty if and only if $P$ has operations of arity $0$.



For $A\xto{f} I \in \Set/I$, we define $P_A$ to be the polynomial endofunctor
\[
  I\xot{s} E\xto{p+\emptyset} B+A \xto{(t,f)} I.
\]


\begin{remark}
  For $A\xto{f} I$, let $A$ denote the polynomial endofunctor $I\ot\emptyset\to
  A\xto{f} I$. Then $P_A$ is the coproduct $P+A$ in $\PolyEnd_I$.
\end{remark}

\begin{lemma} \label{lemma:free-algebra} For all $A\xto{f}I$, $P_A^\moo\xto{t}I$
  is the free $P$-algebra on $A\xto{f}I$.
\end{lemma}
\begin{proof}
  An element of $P(P_A^\moo)$ is by definition a formal composite $b(x_i)_{i\in
    p\inv b}$ of an operation $b$ of $P$ along with an element $x_i\in P_A^\moo$
  for each input of $b$. But this is clearly once again an element of
  $P_A^\moo$, i.e. a leafless tree of operations of $P_A$. Thus we have a
  canonical function $P(P_A^\moo)\xto{\alpha_A} P_A^\moo$ in $\Set/I$,
  exhibiting $P_A^\moo$ as a $P$-algebra. It is easy to see that $\alpha_A$ is
  an inclusion, whose complement is just the subset $A\subset P_A^\moo$.

  Now if $P(X)\xto{\alpha} X$ is any $P$-algebra, then a function $g:A\to X$ in
  $\Set/I$ gives rise to an algebra morphism $\bar{g}:P_A^\moo\to X$ defined
  inductively as
  \begin{align*}
    \bar{g} :  \quad a &\mapsto g(a),\quad a\in A \\
    b(x_i) &\mapsto \alpha_A(b(\bar{g}(x_i)))
  \end{align*}
  and it is easy to see that restriction to $A \subset P_A^\moo$ defines an
  inverse to the function $(\bar{\ }):Set/I(A,X)\to P\alg(P_A^\moo,X)$.
\end{proof}

\begin{corollary}
  $P^\moo$ is the initial object of $P\alg$.
\end{corollary}
\begin{proof}
  Clearly, $P^\moo=P_\emptyset^\moo$, which is the free algebra on $\emptyset\to
  I$ by the previous lemma. The free-algebra functor is a left adjoint, and we
  conclude since $\emptyset\to I$ is initial in $\Set/I$.
\end{proof}


For a more conceptual explanation, note that lemma \ref{lemma:free-algebra} is
equivalent to the following exercise.
\begin{exercise}
  For $P\in\PolyEnd_I$ and $A\xto{f} I$, we have the equivalence of categories
  $P_A\alg\simeq P_A^\moo/P\alg$, where $P_A^\moo/P\alg$ denotes the coslice
  category under $P_A^\moo$.
\end{exercise}


Remark that the function symbols (operations) of the signature $\underline{P} =
(B,I,I,p\inv,s,t)$ can be \emph{composed} in a meaningful way, since the set of
output types is the same as the set of input types. This information is captured
in the definition of the \emph{term algebra} on a signature.
\begin{definition}
  Let $S=(B,I,I,p\inv,s,t)$ be a signature. Then the \emph{typed term algebra}
  of $S$ is the set $S^*$ of trees inductively defined by the grammar
  \[
    (i\in I,b\in B)\quad u^i ::= x^i \ssep (b(u_e^{s(e)})_{e\in p\inv b})^{t(b)}
  \]
  where each $x^i$ is a variable of type $i\in I$.

  A \emph{closed term} is an element of $S^*$ that does not contain any
  variables.
\end{definition}

\begin{remark}
  We immediately note the canonical bijection $(\underline{P})^* \cong
  \bigcup_{n\in\NN} \{ T\in P^n \}$ of sets.
\end{remark}

\begin{remark}
  $P^\moo$ is precisely the set of closed terms in $(\underline{P})^*$.
\end{remark}

For $A \xto{f} I$, it is easy to see that the signature $\underline{P_A}$ is
obtained by adding an operation of arity $0$ to the signature $\underline{P}$
for each element of $A$, with output type given by $f$.

\subsection{The ``free-polynomial-monad'' monad}
Let the category of \emph{polynomial monads} $\PolyMnd_I$ be the category of
monoids in the monoidal category $\PolyEnd_I$. Its objects are therefore monads
on $\Set/I$ whose endofunctor is polynomial, and whose unit and multiplication
are cartesian natural transformations, and we have the obvious square of
forgetful functors

\begin{displaymath}
  \begin{tikzcd}
    \PolyEnd_I \ar[d]
    &\PolyMnd_I \ar[l]\ar[d]\\
    \fun {\Set/I}{\Set/I}
    &\Mnd(Set/I). \ar[l]
  \end{tikzcd}
\end{displaymath}

% \[
%   \xymatrix{ \PolyEnd_I\ar[d]
%     &\PolyMnd_I\ar[l]\ar[d]\\
%     [\Set/I,\Set/I] &\Mnd(Set/I).\ar[l] }
% \]

Theorem \ref{thm:free-monad} now tells us that the endofunctor
$P^*:A\mapsto P_{(A)}^\moo$ is the free monad on the polynomial endofunctor $P$.
It is easy to show that the construction is functorial on $P$, and therefore
defines a functor $(-)^*:\PolyEnd_I\to\Mnd(Set/I)$
This subsection is devoted to showing that the functor $(-)^*$ factors through
$\PolyMnd_I$, and thus provides a left adjoint to the forgetful functor
$\PolyMnd_I\to\PolyEnd_I$, living above $\Mnd(\Set/I)\to[\Set/I,\Set/I]$. 

\begin{lemma}
    Let $S=$
\end{lemma}

\subsection{Term algebras}



%%% Local Variables:
%%% mode: latex
%%% TeX-master: "../notes"
%%% End:
