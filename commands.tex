%%	MATHS

\newcommand{\inft}{{$\infty$}}
\newcommand{\oo}{{$\infty$}} % For use in text mode
\newcommand{\moo}{{\infty}}  % For use in math mode

%	bold numbers
\newcommand{\bzero}{{\mathbf 0 }}
\newcommand{\bone}{{\mathbf 1 }}
\newcommand{\btwo}{{\mathbf 2 }}

%	blackboard bold numbers
\newcommand{\bbzero}{{\mathbbm 0 }}
\newcommand{\bbone}{{\mathbbm 1 }}
\newcommand{\bbtwo}{{\mathbbm 2 }}

%	underlined numbers
\newcommand{\uzero}{{\underline 0 }}
\newcommand{\uone}{{\underline 1 }}
\newcommand{\utwo}{{\underline 2 }}
\newcommand{\uthree}{{\underline 3 }}


%	Greek capitals
\newcommand{\Alpha}{\mathrm{A}}
\newcommand{\Beta}{\mathrm{B}}
\newcommand{\Kappa}{\mathrm{K}}



%	blackboard bold alphabet
\renewcommand{\AA}{\mathbb{A}}
\newcommand{\BB}{\mathbb{B}}
\newcommand{\CC}{\mathbb{C}}
\newcommand{\DD}{\mathbb{D}}
\newcommand{\EE}{\mathbb{E}}
\newcommand{\FF}{\mathbb{F}}
\newcommand{\GG}{\mathbb{G}}
\newcommand{\HH}{\mathbb{H}}
\newcommand{\II}{\mathbb{I}}
\newcommand{\JJ}{\mathbb{J}}
\newcommand{\KK}{\mathbb{K}}
\newcommand{\LL}{\mathbb{L}}
\renewcommand{\L}{\mathrm{L}}
\newcommand{\MM}{\mathbb{M}}
\newcommand{\NN}{\mathbb{N}}
\newcommand{\OO}{\mathbb{O}}
\newcommand{\PP}{\mathbb{P}}
\newcommand{\QQ}{\mathbb{Q}}
\newcommand{\RR}{\mathbb{R}}
\renewcommand{\SS}{\mathbb{S}}
\newcommand{\TT}{\mathbb{T}}
\newcommand{\UU}{\mathbb{U}}
\newcommand{\VV}{\mathbb{V}}
\newcommand{\WW}{\mathbb{W}}
\newcommand{\XX}{\mathbb{X}}
\newcommand{\YY}{\mathbb{Y}}
\newcommand{\ZZ}{\mathbb{Z}}

%	script alphabet
\newcommand{\cA}{\mathcal{A}}
\newcommand{\cB}{\mathcal{B}}
\newcommand{\cC}{\mathcal{C}}
\newcommand{\cD}{\mathcal{D}}
\newcommand{\cE}{\mathcal{E}}
\newcommand{\cF}{\mathcal{F}}
\newcommand{\cG}{\mathcal{G}}
\newcommand{\cH}{\mathcal{H}}
\newcommand{\cI}{\mathcal{I}}
\newcommand{\cJ}{\mathcal{J}}
\newcommand{\cK}{\mathcal{K}}
\newcommand{\cL}{\mathcal{L}}
\newcommand{\cM}{\mathcal{M}}
\newcommand{\cN}{\mathcal{N}}
\newcommand{\cO}{\mathcal{O}}
\newcommand{\cP}{\mathcal{P}}
\newcommand{\cQ}{\mathcal{Q}}
\newcommand{\cR}{\mathcal{R}}
\newcommand{\cS}{\mathcal{S}}
\newcommand{\cT}{\mathcal{T}}
\newcommand{\cU}{\mathcal{U}}
\newcommand{\cV}{\mathcal{V}}
\newcommand{\cW}{\mathcal{W}}
\newcommand{\cX}{\mathcal{X}}
\newcommand{\cY}{\mathcal{Y}}
\newcommand{\cZ}{\mathcal{Z}}

%	bold aphabet
\newcommand{\bA}{\mathbf{A}}
\newcommand{\bB}{\mathbf{B}}
\newcommand{\bC}{\mathbf{C}}
\newcommand{\bD}{\mathbf{D}}
\newcommand{\bE}{\mathbf{E}}
\newcommand{\bF}{\mathbf{F}}
\newcommand{\bG}{\mathbf{G}}
\newcommand{\bH}{\mathbf{H}}
\newcommand{\bI}{\mathbf{I}}
\newcommand{\bJ}{\mathbf{J}}
\newcommand{\bK}{\mathbf{K}}
\newcommand{\bL}{\mathbf{L}}
\newcommand{\bM}{\mathbf{M}}
\newcommand{\bN}{\mathbf{N}}
\newcommand{\bO}{\mathbf{O}}
\newcommand{\bP}{\mathbf{P}}
\newcommand{\bQ}{\mathbf{Q}}
\newcommand{\bR}{\mathbf{R}}
\newcommand{\bS}{\mathbf{S}}
\newcommand{\bT}{\mathbf{T}}
\newcommand{\bU}{\mathbf{U}}
\newcommand{\bV}{\mathbf{V}}
\newcommand{\bW}{\mathbf{W}}
\newcommand{\bX}{\mathbf{X}}
\newcommand{\bY}{\mathbf{Y}}
\newcommand{\bZ}{\mathbf{Z}}
\newcommand{\bCat}{\mathbf{Cat}}

%	rm aphabet
\newcommand{\rmA}{\mathrm{A}}
\newcommand{\rmB}{\mathrm{B}}
\newcommand{\rmC}{\mathrm{C}}
\newcommand{\rmD}{\mathrm{D}}
\newcommand{\rmE}{\mathrm{E}}
\newcommand{\rmF}{\mathrm{F}}
\newcommand{\rmG}{\mathrm{G}}
\newcommand{\rmH}{\mathrm{H}}
\newcommand{\rmI}{\mathrm{I}}
\newcommand{\rmJ}{\mathrm{J}}
\newcommand{\rmK}{\mathrm{K}}
\newcommand{\rmL}{\mathrm{L}}
\newcommand{\rmM}{\mathrm{M}}
\newcommand{\rmN}{\mathrm{N}}
\newcommand{\rmO}{\mathrm{O}}
\newcommand{\rmP}{\mathrm{P}}
\newcommand{\rmQ}{\mathrm{Q}}
\newcommand{\rmR}{\mathrm{R}}
\newcommand{\rmS}{\mathrm{S}}
\newcommand{\rmT}{\mathrm{T}}
\newcommand{\rmU}{\mathrm{U}}
\newcommand{\rmV}{\mathrm{V}}
\newcommand{\rmW}{\mathrm{W}}
\newcommand{\rmX}{\mathrm{X}}
\newcommand{\rmY}{\mathrm{Y}}
\newcommand{\rmZ}{\mathrm{Z}}

%	sf aphabet
\newcommand{\sfA}{\mathsf{A}}
\newcommand{\sfB}{\mathsf{B}}
\newcommand{\sfC}{\mathsf{C}}
\newcommand{\sfD}{\mathsf{D}}
\newcommand{\sfE}{\mathsf{E}}
\newcommand{\sfF}{\mathsf{F}}
\newcommand{\sfG}{\mathsf{G}}
\newcommand{\sfH}{\mathsf{H}}
\newcommand{\sfI}{\mathsf{I}}
\newcommand{\sfJ}{\mathsf{J}}
\newcommand{\sfK}{\mathsf{K}}
\newcommand{\sfL}{\mathsf{L}}
\newcommand{\sfM}{\mathsf{M}}
\newcommand{\sfN}{\mathsf{N}}
\newcommand{\sfO}{\mathsf{O}}
\newcommand{\sfP}{\mathsf{P}}
\newcommand{\sfQ}{\mathsf{Q}}
\newcommand{\sfR}{\mathsf{R}}
\newcommand{\sfS}{\mathsf{S}}
\newcommand{\sfT}{\mathsf{T}}
\newcommand{\sfU}{\mathsf{U}}
\newcommand{\sfV}{\mathsf{V}}
\newcommand{\sfW}{\mathsf{W}}
\newcommand{\sfX}{\mathsf{X}}
\newcommand{\sfY}{\mathsf{Y}}
\newcommand{\sfZ}{\mathsf{Z}}

%%	MISCELLANEOUS
\newcommand{\inv}{^{-1}}					% exponential notation for the inverse of something

\newcommand{\restr}[1]{{_{\mkern 1mu \vrule height 2ex\mkern2mu #1}}}


%%	CATEGORY THEORY



%	Arrows

\newcommand{\lto}{\leftarrow}					% short left arrow
\newcommand{\ot}{\leftarrow}					% short left arrow

\newcommand{\tto}{\longrightarrow}				% long right arrows
\newcommand{\mto}{\longmapsto}				% long maps to

\newcommand{\subto}{\hookrightarrow}			% subobject
\newcommand{\mono}{\rightarrowtail}			% monomorphisms
\newcommand{\epi}{\twoheadrightarrow}			% epimorphisms
\newcommand{\surj}{\twoheadrightarrow}			% surjections
\newcommand{\cover}{\twoheadrightarrow}		% cover


\newcommand{\Ra}{\Rightarrow}				% double right arrows
\newcommand{\RA}{\Longrightarrow}			% long double right arrows
\newcommand{\QRA}{\quad\Longrightarrow\quad}	% long double right arrows with spaces
\newcommand{\La}{\Leftarrow}					% double left arrows
\newcommand{\LA}{\Longleftarrow}				% long double right arrows

\renewcommand{\iff}{\Leftrightarrow}				% short logical equivalence
\newcommand{\IFF}{\Longleftrightarrow}			% long logical equivalence

\newcommand{\qiff}{\quad\Leftrightarrow\quad}				% short logical equivalence with spaces
\newcommand{\QIFF}{\quad\Longleftrightarrow\quad}			% long logical equivalence with spaces



\newcommand{\xto}[1]{\xrightarrow {#1}}			% variable size right arrow
\newcommand{\xlto}[1]{\xleftarrow {#1}}			% variable size left arrow
\newcommand{\xot}[1]{\xleftarrow {#1}}			% variable size left arrow



%	Diagrams



% Pullback and pushout marks for diagrams
% for use with tikz-cd
\newcommand{\pbmark}{\ar[dr, phantom, "\ulcorner" very near start, shift right=1ex]}
\newcommand{\pomark}{\ar[ul, phantom, "\lrcorner" very near start, shift right=1ex]}

% variation with two columns jump
\newcommand{\pbmarkk}{\ar[drr, phantom, "\ulcorner" very near start, shift right=1ex]}
\newcommand{\pomarkk}{\ar[ull, phantom, "\lrcorner" very near start, shift right=1ex]}

% for use with xy
\newcommand{\pullbackcorner}[1][ul]{\save*!/#1+1.8pc/#1:(1.5,-1.5)@^{|-}\restore}
\newcommand{\pushoutcorner}[1][ul]{\save*!/#1-1.5pc/#1:(-1,1)@^{|-}\restore}


% Adjunctions

% Usage : \adj{1}{2}{3}{4} where 1-left category, 2-right category,
% 3-left adjoint label, 4-right adjoint label
\newcommand{\adj}[4]{#3 : #1 \leftrightarrow #2 : #4 } 


%	Hom spaces & core

\newcommand{\Hom}[2]{{\textrm{Hom}\!\left(#1,#2\right)}}			% Hom
\newcommand{\HOM}{\textrm{Hom}}							% name of Hom functor
%\newcommand{\relHom}[3]{{\textrm{Hom}_{#1}\!\left(#2,#3\right)}}	% Hom in category #1
%\newcommand{\relHOM}[1]{{\textrm{Hom}_{#1}}}				% name of Hom functor in category #1

\newcommand{\relHom}[3]{{{#1}\!\left(#2,#3\right)}}				% Hom in category #1 denoted with name of category


\newcommand{\End}[1]{{\textrm{End}\left(#1\right)}}				% basic endomorphism set

\newcommand{\Map}[2]{{\textrm{Map}\left(#1,#2\right)}}			% basic mapping space

\newcommand{\map}[2]{\left[#1,#2\right]}						% external hom between #1 and #2
\newcommand{\relmap}[3]{\left[#2, #3\right]_{#1}}				% external hom  between #2 and #3 relative to #1

\newcommand{\intmap}[2]{\left\lsem #1, #2 \right\rsem}			% internal hom between #1 and #2
\newcommand{\relintmap}[3]{\left\lsem #2, #3 \right\rsem_{#1}}		% internal hom between #2 and #3 relative to #1

\newcommand{\core}[1]{{\textrm{core}\!\left(#1\right)}}			% core of category #1

\newcommand{\alg}{{\text{-} \mathrm{\cA lg}}} % category of algebras


%	source and target functors
%\renewcommand{\r}{\mathrm{r}}
\newcommand{\s}{\mathrm{s}}								% source functor
\renewcommand{\t}{\mathrm{t}}								% target functor


%	Limits and colimits
\DeclareMathOperator*{\colim}{colim}						% colimit operator
\DeclareMathOperator*{\fcolim}{fin-colim}						% finite colimit operator

\DeclareMathOperator*{\flim}{fin-lim}							% finite limit operator

\newcommand{\cst}{\mathrm{cst}}							% notation of the constant diagram functor


%	Dependent sums and products
\newcommand{\depsum}[2]{\sum_{#1}#2}						% dependent sum
\newcommand{\depprod}[2]{\prod_{#1}#2}						% dependent product



%	Notations for functor categories
\newcommand{\fun}[2]{\left[#1,#2\right]}						% functor category
\newcommand{\Mnd}{{\mathrm{\cM nd}}}

\newcommand{\arr}{^\rightarrow}							% arrow category
\newcommand{\iso}{^\simeq}								% category of isomorphism

\newcommand{\op}{^{op}}									% opposite category
\newcommand{\opone}{^{1op}}								% opposite of 1-arrow for 2-categories
\newcommand{\optwo}{^{2op}}								% opposite of 2-arrow for 2-categories
\newcommand{\oponetwo}{^{1op,\, 2op}}						% opposite of 2-arrow for 2-categories


\newcommand{\cc}{_{cc}}									% cocomplete / cocontinuous / cocompletion
\renewcommand{\c}{^{c}}									% complete / continuous / limit completion
\newcommand{\cont}{^{c}}									% same

\newcommand{\bicc}{_{cc,\, cc}}							% bi cocontinuous functors of two variables
\newcommand{\bic}{^{c,\, c}}								% bi continuous functors of two variables
\newcommand{\bilex}{^{lex,\, lex}}							% bi lex functors of two variables


\newcommand{\filt}{_{filt}}							% filtered cocomplete / filtered cocontinuity / filtered cocompletion notation
\newcommand{\cfilt}{^{filt}}						% filtered complete / filtered continuity / filtered completion notation

\newcommand{\ind}{_{ind}}						% ind index
\newcommand{\pro}{^{pro}}						% pro exponent



\newcommand{\rex}{_{rex}}								% finitely cocomplete / finite cocontinuity / finite cocompletion
\newcommand{\lex}{^{lex}}								% finitely complete / finite continuity / finite completion
\renewcommand{\flat}{^{flat}}								% flat functor notation

\newcommand{\pointed}{^\bullet}							% exponent for a pointed object
\newcommand{\pointedexp}[1]{{^{\bullet #1}}}					% pointed exponential

\newcommand{\slice}[1]{_{/#1}}								% slice
\newcommand{\coslice}[1]{_{#1/}}							% coslice



%	Some categories


\newcommand{\Ens}{{\mathcal{E} ns}}						% categorie des ensembles
\newcommand{\Set}{{\mathcal{S} et}}						% category of sets
\newcommand{\SSet}{\mathrm{SSet}}							% category of simplicial sets

\newcommand{\PreOrder}{\mathrm{\mathcal{P} re\mathcal{O} rder}}		% category of pre-orders

\newcommand{\Top}{\mathrm{\mathcal{T} op}}						% category of topological spaces

\newcommand{\Fin}{\mathrm{ \mathcal{F} in}}						% category of finite sets / spaces
\newcommand{\Finp}{{\mathrm{\mathcal{F} in}^\bullet}}				% category of finite pointed sets / spaces
\newcommand{\Fino}{{\mathrm{\mathcal{F} in}^\circ}}				% category of finite non-empty / spaces

\newcommand{\Gp}{\mathcal{G}{p}}								% category of groups
\newcommand{\Ab}{\mathcal{A}{b}}                % category of abelian groups

\newcommand{\CRing}{{\mathcal{CR}ing}}
\newcommand{\Mod}{\text{-Mod}}                 % for category of modules e.g. $A\Mod$

\newcommand{\Pos}{\mathcal{P}{os}}             % category of partial orders

\newcommand{\Cat}{\mathrm{\mathcal{C} at}} 						% category of small categories
\newcommand{\CAT}{\widehat{ \mathrm{\mathcal{C} at}} }				% category of normal categories

\newcommand{\Pres}{\mathrm{\mathcal{P} res}} 					% category of smally presentable normal categories
\newcommand{\Psh}[1]{\mathrm{\mathcal{P} rshv(#1)}} 					% category of presheaves categories
\newcommand{\psh}[1]{\widehat{#1}}


\newcommand{\Sh}[1]{{\mathrm{\mathcal{S} h}\!\left(#1\right)}}			% category of sheaves over #1
\newcommand{\Shv}[1]{{\mathrm{\mathcal{S} hv}\!\left(#1\right)}}		% alternate version

\newcommand{\Ind}[1]{{\mathrm{\mathcal{I} nd}\!\left(#1\right)}}				% ind-completion of #1
\newcommand{\IND}{\mathrm{\mathcal{I} nd}}							% name of the functor of ind-completion
\newcommand{\Indk}[2]{{\mathrm{\mathcal{I} nd}_{#1}\!\left(#2\right)}}		% #1-ind-completion of #2
\newcommand{\INDk}[1]{{\mathrm{\mathcal{I} nd}_{#1}}}					% name of the functor of #1-ind-completion

\newcommand{\Pro}[1]{{\mathrm{\mathcal{P} ro}\!\left(#1\right)}}			% pro-completion of #1
\newcommand{\PRO}{\mathrm{\mathcal{P} ro}}							% name of the functor of pro-completion
\newcommand{\Prok}[2]{{\mathrm{\mathcal{P} ro}_{#1}\!\left(#2\right)}}		% #1-pro-completion of #2
\newcommand{\PROk}[1]{{\mathrm{\mathcal{P} ro}_{#1}}}					% name of the functor of #1-pro-completion








